\documentclass{article}

\usepackage{fancyhdr} % Required for custom headers
\usepackage{lastpage} % Required to determine the last page for the footer
\usepackage{extramarks} % Required for headers and footers
\usepackage{graphicx} % Required to insert images

% Margins
\topmargin=-0.45in
\evensidemargin=0in
\oddsidemargin=0in
\textwidth=6.5in
\textheight=9.0in
\headsep=0.25in 


\linespread{1.1} % Line spacing

% Set up the header and footer
\pagestyle{fancy}
\lhead{\hmwkAuthorName} % Top left header
\chead{\hmwkTitle} % Top center header
\rhead{\firstxmark} % Top right header
\lfoot{\lastxmark} % Bottom left footer
\cfoot{} % Bottom center footer
\rfoot{Page\ \thepage\ of\ \pageref{LastPage}} % Bottom right footer
\renewcommand\headrulewidth{0.4pt} % Size of the header rule
\renewcommand\footrulewidth{0.4pt} % Size of the footer rule

\setlength\parindent{0pt} % Removes all indentation from paragraphs

%----------------------------------------------------------------------------------------
%	DOCUMENT STRUCTURE COMMANDS
%	Skip this unless you know what you're doing
%----------------------------------------------------------------------------------------



%----------------------------------------------------------------------------------------
%	NAME AND CLASS SECTION
%----------------------------------------------------------------------------------------

\newcommand{\hmwkTitle}{Star Formation History Information Content of Spectra:  Cramer-Rao Bound} % Assignment title
\newcommand{\hmwkDueDate}{} % Due date
\newcommand{\hmwkClass}{} % Course/class
\newcommand{\hmwkClassTime}{} % Class/lecture time
\newcommand{\hmwkClassInstructor}{} % Teacher/lecturer
\newcommand{\hmwkAuthorName}{B Johnson} % Your name




\begin{document}

\vspace{0.15in}

\underline{\textsc{Introduction}}: 
Knowledge of the star formation history of galaxies of differnt types at different redshifts would place strong constraints on theories of galaxy evolution.
However, it is difficult to infer star formation histories on a galaxy by galaxy basis from the available observational data.
Probably the most informative exisiting data are color magnitude diagrams of the stars comprising a galaxy.
These data are expensive to collect, and its only possible for very nearby galaxies.
The next most informative is probably integrated or integral field spectroscopy of galaxies.
This data is far more plentiful.
But we don't really know how well it tells us things....

\underline{\textsc{Basics}}:
We define SFH as the function 
\begin{eqnarray}
\mathrm{SFH} & \equiv & \psi(t, Z)
\end{eqnarray}
where $t$ is the lookback time,
$Z$ is the metallicity,
and $\psi$ is the mass of stars formed per unit lookback time and unit metallicity.
Note that we use lookback time since it avoids some annoying axis reversals and simplifies notation.

We use the basic equation of stellar population synthesis to describe the integrated spectrum $F_\lambda$ at time $T$ in terms of the SFH and other parameters.
\begin{eqnarray}
F_\lambda & = & \int_0^{t_{univ}} dt \, \int_{Z_{min}}^{Z_{max}} dZ \, \psi(t, Z) \, s_\lambda(t, Z) \, e^{-\tau_\lambda(t, Z)}
\end{eqnarray}
where $s_\lambda(t, Z)$ is the spectrum of a unit mass of stars of age $t$ and metallicity $Z$, and
$\tau_\lambda(t, Z)$ is the \emph{effective} dust opacity towards stars of age $t$ and metallicity $Z$.

In practice, this is approximated by a sum over disrete \emph{simple stellar populations} (SSPs).  
These SSPs are the $s_\lambda(t, Z)$ for $J$ specific values of $t$ (indexed as $t_j$) and $K$ specific values $Z$ (indexed as $Z_k$).
They are calculated using isochrones and stellar spectral libraries.
The sum is then
\begin{eqnarray}
F_\lambda & = & \sum_{j=1}^J \, \sum_{k=1}}^{K} \, m_{j,k} \, s_{\lambda, j,k} \, e^{-\tau_{j,k}} \\
s_{\lambda, j,k} & = & s_\lambda(t_j, Z_k)
\end{eqnarray}
where the $m_{j,k}$ are the masses or \emph{weights} of each SSP.  
These weights are determined by $\psi(t, Z)$ and the SSP interpolation scheme,  and are constrained to be positive.
They also fully specify the SFH.

If we collapse the two indices $j, k$ into a single index $i \equiv j\cdot K + k$ and ignore dust then we can write this in matrix form
\begin{eqnarray}
F & = & M \, S
\end{eqnarray}
Assuming the SSP spectra $s_{i}$ to be linearly independent, $L < J\times K$, perfect knowledge of $F$ (the integrated spectrum), and no dust, 
then this equation can be solved using standard linear algebra techniques.

However, we do not know $F$ perfectly, and there is dust. 
The question then is, given $L_\lambda$ to some known precision, how well can we constrain the $m_{j,k}$ values? 
Alternatively, what signal to noise is required to constrain $m_{j,k}$ to a specified precision. 
Another interesting question concerns the number of independent $m_{j,k}$ values that can be constrained to a given precision -- 
that is, the time (and metallicity) resolution of the SFH that is acheivable for a given integrated spectrum.
A clear and systematic answer to this question has not yet been given, though much excellent work in this direction has been done.


\underline{\textsc{The Cramer-Rao bound}}:
A complete answer is actually quite difficult, as one must somehow explore the full joint posterior probability distribution for the $m_{j,k}$, 
which in general can be complex and depends on the SFH itself.
Efforts have been made using non-negative matrix factorization of Monte Carlo samplings of the noise.

However, we can obtain \emph{limits} on the acheivable accuracy.  
We will do this using the Fisher Information Matrix and the associated Cramer-Rao bound.
The Cramer- Rao bound is based on estimating the curvature of the posterior PDF from the derivatives of the likelihood (or posterior probability) function.
It thus requires us to write down a likelihood and take its derivative.

For simplicity, we are going to restrict ourselves (for the moment) to the case of no dust and a single metallicity, $K=1$.
We write the likelihood as 
\begin{eqnarray}
\mathcal{L} & = & 
\end{eqnarray}


\end{document}